%%%%%%%%%%%%%%%%%%%%%%%%%%%%%%%%%%%%%%START PREAMBLE THAT IS THE SAME FOR ALL EXAMPLES
\documentclass{article}

%Required: You must have these
\usepackage{Sweave}
\usepackage{graphicx}
\usepackage{tabularx}
\usepackage{hyperref}
\usepackage{natbib}
\usepackage{pdflscape}
\usepackage{array}
\usepackage{gensymb}
%\usepackage[backend=bibtex]{biblatex}
%Strongly recommended
  %put your figures in one place
%\SweaveOpts{prefix.string=figures/, eps=FALSE} 
%you'll want these for pretty captioning
\usepackage[small]{caption}

\setkeys{Gin}{width=0.8\textwidth}  %make the figs 50 perc textwidth
\setlength{\captionmargin}{30pt}
\setlength{\abovecaptionskip}{10pt}
\setlength{\belowcaptionskip}{10pt}
% manual for caption  http://www.dd.chalmers.se/latex/Docs/PDF/caption.pdf

%Optional: I like to muck with my margins and spacing in ways that LaTeX frowns on
%Here's how to do that
 \topmargin -1.5cm        
 \oddsidemargin -0.04cm   
 \evensidemargin -0.04cm  % same as oddsidemargin but for left-hand pages
 \textwidth 16.59cm
 \textheight 21.94cm 
 %\pagestyle{empty}       % Uncomment if don't want page numbers
 \parskip 7.2pt           % sets spacing between paragraphs
 %\renewcommand{\baselinestretch}{1.5} 	% Uncomment for 1.5 spacing between lines
\parindent 0pt% sets leading space for paragraphs
\usepackage{setspace}
%\doublespacing

%Optional: I like fancy headers
%\usepackage{fancyhdr}
%\pagestyle{fancy}
%\fancyhead[LO]{How do climate change experiments actually change climate}
%\fancyhead[RO]{2016}
 
%%%%%%%%%%%%%%%%%%%%%%%%%%%%%%%%%%%%%%END PREAMBLE THAT IS THE SAME FOR ALL EXAMPLES

%Start of the document
\begin{document}

%\SweaveOpts{concordance=TRUE}

\bibliographystyle{../refs/bibstyles/amnat.bst}% i moved a style file into the ospree git repo. feel free to add whatever style you like and update, lizzie! I don't have besjournals
\title{Shifts in Southern Resident Killer Whale Phenology in the Salish Sea}
\date{\today}
\maketitle
\author{A.K. Ettinger, C. Harvey, J. Samhouri, B. Hanson, C. Emmons, J. Olson, E. Ward}
\maketitle  %put the fancy title on
%%%%%%%%%%%%%%%%%%%%%%%%%%%%%%%%%%%%%%%%%%%%%%%%%%%

\section* {Background}
\begin{enumerate}
\item Southern resident killer whales (SRKWs) are a threatened population, recieving much scientific and public attention. Discuss threats to SRKW.
\item  SRKWs use of the Salish Sea varies seasonally across two broad areas: the upper Salish Sea (north of Admiralty Inlet, reference Map) is commonly used in the summer; Puget sound (south of Admiralty Inlet) is most commonly used in the fall and winter.
\item The seasonal patterns in SRKW activity are generally thought to be related to finding prey (citation for this?). In recent decades, salmon abundance and phenology has shifted in the Salish Sea \citep{weinheimer2017,reed2011,ford2006}(add Nelson for chinook hatchery release timing, others for abundance data), though rates vary by species and location. We would therefore expect SRKW phenology to have shifted during this time, if prey is a primary driver of their activity in the Salish Sea. 
\item It has been suggested that seasonal activit of SRKWs may be shifting in recent years. For example, they appear to be spending more time in Puget Sound in recent decades \citep{olson2018}. These charismatic megafauna have been increasingly watching by the general public, however, so monitoring effort has also increased duiring this time. Teasing apart changes in effort from shifts in SRKW activity is important, as understanding how SRKW activity varies seasonally and how these seasonal patterns have shifted in recent decades will allow us to develop and test hypotheses about potential drivers of these shifts. Thus in turn will provide information that may be useful for management decision-making to conserve SRKWs.

%\item Efforts are underway to alter hatchery production to benefit orca whales, because one of the threats facing SRKWs is thought to be lack of prey. We know that SRKWs forage on chinook salmon as a primary food source (cite hanson papers), but there are lots of questions about what other species they use and how their prey varies seasonally. 


\end{enumerate}
\section* {Research Questions}
Here, we ask:
\begin{enumerate}
\item Has the timing of SRKW activity (phenology) shifted in the upper Salish Sea and/or Puget Sound? (First observation date, Last observation date, number of days observed)
\item If there have been phenological shifts in SRKW activity, do these shifts coincide with shifts in phenology or abundance of their prey (chinook, coho, chum salmon)?
\end{enumerate}


\section* {Methods}
\begin{enumerate}
\item Data
\begin{enumerate}
\item To quantify orca seasonal phenology over time, we used the OrcaMaster Database for Whale Sighting Data (Whale Museum), comprised of data from five main sources, including public sightings networks (e.g., OrcaNetwork), commercial whale watch data, and scientific surveys (e.g., SPOT data from satellite tracking units) \citep{olson2018}. We used data from 1978-2017, because prior to 1976, there was no dedicated effort to track SRKW presence in the region \citep{olson2018}. We used these sighting data to identify first observation day-of-year (DOY, the first date on which one or more pods of SRKWs were observed) and last observation DOY (the first date on which one or more pods of SRKWs were observed) for each sub-region (Upper Salish Sea and Puget Sound) and season (Winter-- October through January; and Summer-- May through September). We used these seasonal definitions because they are most aligned with mean SRKW seasonal activity patterns over time (Fig \ref{fig:phenplot},Fig \ref{fig:phenplot2}).


\item WDFW adult salmon return data for coho, chum, chinook in XX streams (Alternatively, may use RMIS data for coho, chum, chinook.)
\end{enumerate}
\item Analysis
\begin{enumerate}
\item To identify trends over time in phenology for all SRKWs in the Upper Salish Sea and in Puget Sound, we used linear regression on first  and last observation dates from 1978 through 2017. We also compared first and last observation dates across time periods (1978-1997 versus 1998-2018), using boxplots and t-tests (Fig \ref{fig:boxplot1} and \ref{fig:boxplot2}), because relationships over time did not look linear and linear regression provided poor fit (Fig \ref{fig:reg}). 
\item To estimate effects of increased effort (i.e., increased numbers of sightings over time) on trends in phenology, we simulated datasets of whale presence during two seasons equivalent to those in our dataset (summer, which was 1 May through 31 Sept, or 153 days, and fall, which was 1 October through 1 Feb, or 123 days). We used whale presence probabilities of 0.85 for the Upper Salish Sea and 0.6 for Puget Sound (the means in our dataset for each region) and kept it constant over 40 simulated years. We then created an observation dataset, in which effort (the number of observations), varied. During the low effort time period (years 1-20), the number of observations had a mean of 15 per year for Puget Sound and 104 per year in the Upper Salish Sea (matching the means for these reagions from 1978-1997 in the OrcaMaster database).  During the high effort time period (years 21-40 in our simulated dataset), the number of annual observations had a mean of 39 for Puget Sound andand 133 for the Upper Salish Sea (matching those in the OrcaMAster database from 1998-2017). We then calculated first- and last- observations dates for each simulated year. We ran these simulation 100 times and calculated the difference between the low effort and high effort time periods. We compared these to the mean differences first- and last-observation daates across time periods in the OrcaMaster database, for each region, to understand whether observed changes may be due to changes in effort over time, rather than changes in orca activity (Fig \ref{fig:sims}).

\item To identify trends over time in phenology for each pod separately (J,K,L) in the Upper Salish Sea and in Puget Sound, we used linear regression on first-, last-, and peak detection dates estimated from pod-specific occupancy models. Occupancy models (MacKenzie et al. 2002) can estimate jointly species presence or abundance and detection probability (the probability to detect at least one individual present at site). We parameterized the model with annual occupancy probabilities (i.e., we did not fit a dynamic model, but a multiseason model (Royle and Kery 2007). The distribution submodel distinguishes true presence or absence of pod,p, (z, a latent state) in marine area i in year t, ...add equation here. We assumed zi,t to be a Bernoulli random variable...
We modelled detection probability as a function of year and date, with detectability modelled as a semi-parametric, smooth function of date using flexible thin-plate spline regression modelling \citep{strebel2014}. 

\item We used linear regression to identify trends over time in first, median, and last dates of salmon adult migration timing.
\end{enumerate}
\end{enumerate}


\section*{Results}
\section*{Discussion}
\section*{Conclusion}

\section* {Figures}
\begin{enumerate}
\item 3-paneled figure with the following panels: 
\begin{enumerate}
\item Map of Salish Sea showing 2 seasons and habitat use areas.
\item Observations in Puget Sound (fall/winter) (rough versions are Fig. \ref{fig:phenplot} and \ref{fig:phenplot2}
\item Observations in the upper salish sea (spring/summer)
\end{enumerate}
\item 4- or 6-paneled figure with trends in first and last observation dates (and number of whale days?) of whales in upper Salish Sea (summer habitat) and south Puget Sound (winter habitat)
\item 6-paneled figured of Pod-specific trends in first and last observation dates puget sound and the upper salish sea (rough version is Fig. \ref{fig:trends}
\end{enumerate}

\begin{figure}[p]
\includegraphics{../analyses/figures/OrcaPhenPlots/SRKW_yr_doy.pdf} 
\caption{\textbf{Southern resident killer whale activity in the Upper Salish Sea and Puget Sound varies by season}. Need to add a map of where observations are, and make this look prettier...}
 \label{fig:phenplot}
 \end{figure}
 
%  \begin{figure}[p]
% \includegraphics{../analyses/figures/basicplots/propwhdaysperweek_std.pdf} 
% \caption{\textbf{Southern resident killer whale observations, in terms of proportion of days per week that whales were observed, in the Outer Coast, Puget Sound and Upper Salish Sea, varies by season}. Proportions are the mean across all years (1978-2017) and have been standardized by subtracting the mean and dividing by the standard deviation. Thus, the "0" line marks the mean proportion for that region. Weeks range from 1 (i.e., the first week in January) to 52 (i.e., the last week in December).}
%  \label{fig:phenplot1}
%  \end{figure}

\begin{figure}[p]
\includegraphics{../analyses/figures/basicplots/propwhdaysperweek.pdf} 
\caption{\textbf{Southern resident killer whale activity varies by season and region}. Lines show proportion of days per week (1 Jan= week 1) in which whales were observed in the region.}
 \label{fig:phenplot2}
 \end{figure} 
\begin{figure}[p]
\includegraphics{../analyses/orcaphen/figures/boxplots_1978_2017_ps.pdf} 
\caption{\textbf{Change in in first and last observation dates of SRKW in Puget Sound.} }
 \label{fig:boxplot1}
 \end{figure}
 \begin{figure}[p]
\includegraphics{../analyses/orcaphen/figures/boxplots_1978_2017_uss.pdf} 
\caption{\textbf{Change in in first and last observation dates of SRKW in the Upper Salish Sea.} }
 \label{fig:boxplot2}
 \end{figure}
 
\begin{figure}[p]
\includegraphics{../analyses/orcaphen/figures/simvdata_1978_2017.pdf} 
\caption{\textbf{Change in first and last observation dates in Puget Sound and the Upper Salish Sea.} Observed changes (black dots) are compared with expected changes due to shifts in effort alone (gray lines show 90 quantiles of from a simulated dataset resulting from shifts in effort, with the probability of SRKW presence was kept constant across time).}
 \label{fig:sim}
 \end{figure}
 

  \begin{figure}[p]
\includegraphics{../analyses/orcaphen/figures/linreg_1978_2017.pdf} 
\caption{\textbf{Time series and linear regression of first and last observation dates of SRKW.} }
 \label{fig:reg}
 \end{figure}
 

 
\begin{figure}[p]
\includegraphics{../analyses/figures/J/orcaphen_1976_2017_PS_winter_Jfirst.pdf} 
\caption{\textbf{Trends in first- and last- observation dates for J, K, and L pod.} Make this a 6-paneled figure}
 \label{fig:trends}
 \end{figure}

\bibliography{../refs/noaalib.bib}
\end{document}
